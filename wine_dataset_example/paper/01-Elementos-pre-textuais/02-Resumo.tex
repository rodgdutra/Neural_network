%=================================================================================================
%=							       		      SUMÁRIO											 =
%=================================================================================================



\begin{center}


\begin{minipage}[c][5cm]{15cm}		%-> [Altura] [Largura]
	
\begin{abstract}

\noindent
O artigo detalha a construção e o uso de 2 estruturas de rede, ambas  aplicadas em um problema de classificação. A primeira estrutura baseia-se no modelo de rede Perceptron de múltiplas camadas(MLP), a outra estrutura é chamada de rede autoassociativa competitiva, também chamada de autoencoder competitiva. Essas estruturas foram aplicadas na classificação de tipos de vinho, visando-se assim comparar o desempenho e acurácia de ambas topologias no problema de classificação proposto.Ambas estruturas foram construídas utilizando software Python juntamente com a biblioteca Keras, o código utilizado na produção dos resultados está disponibilizado no github.  
\\[1 mm]
\textbf{Palavras-chave}: Redes Neurais Artificiais. Redes Auto Associativas. Vinho. MATLAB.

\end{abstract}

\end{minipage}


\end{center} \vspace*{0.5 cm}

