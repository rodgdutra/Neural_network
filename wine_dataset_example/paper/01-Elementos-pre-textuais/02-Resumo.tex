%=================================================================================================
%=							       		      SUMÁRIO											 =
%=================================================================================================



\begin{center}


\begin{minipage}[c][5cm]{15cm}		%-> [Altura] [Largura]
	
\begin{abstract}

\noindent
Nós estudamos duas arquiteturas de Redes Neurais Artificiais para um problema de classificação de três tipos de vinhos, com a finalidade de comparar a acurácia das duas redes no problema de classificação proposto. A primeira arquitetura estudada é uma rede mais simples, a Perceptron de Multiplas Camadas (MPL), e já a segunda é uma Estrutura Competitiva de MPL Auto Associativa. O banco de dados é composto de 178 instâncias de 13 parâmetros das 3 classes de vinho. Primeiramente, nós normalizamos as entradas do banco de dados e implementamos os algorítimos das duas arquiteturas no software MATLAB. Como resultado, podemos identificar através do estudo do erro que a arquitetura de Competição Auto Associativa obteve um melhor desempenho para a classificação do que uma rede Perceptron de Múltiplas Camadas.
\\[1 mm]
\textbf{Palavras-chave}: Redes Neurais Artificiais. Redes Auto Associativas. Vinho. MATLAB.

\end{abstract}

\end{minipage}


\end{center} \vspace*{0.5 cm}

