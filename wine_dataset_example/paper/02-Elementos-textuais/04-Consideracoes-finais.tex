\section{Considerações finais}

\subsection{Discussão}

Ambas estruturas RNA abordadas nesse artigo apresentaram um desempenho bastante satisfatório, com destaque da Rede autoassociativa competitiva que apresentou 100\% de acerto, uma melhora de de quase 5\% comparado com a estrutura MLP padrão. Para tanto, a rede \textit{autoencoders}  competitiva também apresentou um custo computacional mais elevado, com a utilização de 3000 épocas no treinamento e validação de cada 1 das 3 RNA \textit{autoencoders}.

Dessa forma, para treinar redes dedicadas para cada classe, e nesse processo melhorar o desempenho de classificação a rede autocompetitiva foi mais custosa computacionalmente que a MLP padrão.


\subsection{Conclusão}

Com base no problema de classificação proposto, a rede autoassociativa competitiva apresentou um melhor desempenho na classificação dos tipos de vinho que a rede MLP padrão, com a ressalva do tempo de execução e maior número de épocas de treino validação e teste, o que indica que para outras aplicações na qual há uma maior necessidade em relação a rapidez de treino, ou em aplicações na qual não se dispõe de muito poder computacional, como aplicações em sistemas embarcados, a MLP padrão pode-se apresentar como uma solução melhor.

Tendo-se em vista também a quantidades de índex de conjunto de dados do banco de dados analisados, é possível afirmar que o uso de um banco de dados maior poderia acarretar ressaltando o resultado obtido na análise desse artigo.
