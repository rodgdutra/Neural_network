%=================================================================================================
%=							       		    INTRODUÇÃO							     			 =
%=================================================================================================

\section{Introdução}
As Redes Neurais Artificiais (RNA) tratam-se de modelos matemáticos construídos a partir do conhecimento do neurônio biológico, tendo como objetivo mimetizar o comportamento da células nervosa \cite{ref1} . Assim, simulando a característica de um sistema nervoso, de forma a ter capacidade de processar múltiplas entradas, reconhecer e classificar padrões. Dessa forma, RNA tem grande capacidade de aplicação em vários tipos de contexto.

A aplicação proposta nesse artigo trata-se de classificação de tipos de vinho. O banco de dados utilizado foi retirado de \textit{UCI machine learning repository}[2].Esse banco é composto de 179 amostras de 13 atributos de entrada distribuídos em 3 classes, as quais são o tipo de vinho dado a combinação dos valores dos 13  atributos de entrada.

Dessa maneira tendo-se em vista a o conjunto de dados de dados aplicado, foram aplicadas 2 topologias distintas de RNA no problema de classificação proposto, com o intuito de compara-las em termos de acurácia e desempenho. As topologias utilizadas foram rede do tipo MLP e autoassociativa competitiva, as quais serão discutidas e analisadas e caracterizadas mais adiante nesse artigo.


\subsection{Metodologia}

Para o fim proposto, foi usado o Software Python(3.65)[3] juntamente com a biblioteca Keras[4] para gerar as RNA.


\subsection{Objetivos}

O objetivo desse trabalho é primeiramente validar as topologias de redes apresentadas neste, aplicando-as em um problema de classificação.Feito isso, o trabalho também tem como objetivo comparar o desempenho das 2 estruturas de RNA para  o problema proposto.


\subsection{Organização do Trabalho}

A organização do artigo foi estruturado da seguinte forma: Na seção 2 encontra-se a base teórica de RNA e a apresentação dos 2 modelos utilizados; A seção 3 detalha o banco de dados e seu uso dentro do artigo;A seção 4 detalha o desenvolvimento e aplicação da MLP;A seção 5 detalha o desenvolvimento e aplicação da rede autoencoder competitiva;A seção 6 apresenta os resultados; A seção 7 dispoe a conclusão. 