%=================================================================================================
%=							       		    REFERÊNCIAS											 =
%=================================================================================================


\begin{thebibliography}{99}
\bibitem{ref1}
Kovács, Z. Redes neurais artificiais.  (Editora Livraria da Fisic,2002)

\bibitem{ref2}
Blake, C. \& Merz, C. UCI repository of machine learning databases, 1998.

\bibitem{ref3}
Rossum, G. Python reference manual.  (CWI (Centre for Mathematics and Computer Science,1995)

\bibitem{ref4}
Chollet, F. Keras: Deep learning library for theano and tensorflow.  {\em Url: Https://keras.io/k}. \textbf{7}, T1 (201)

\bibitem{ref5}
Haykin, S. Neural networks: a comprehensive foundation.  (Prentice Hall PT,1994)

\bibitem{ref6}
A, V., Castro, A. \& Lima, S. Diagnosing faults in power transformers with autoassociative neural networks and mean shift.  {\em Ieee Transactions On Power Delivery}. \textbf{27}, 1350--1357 (2012)

\bibitem{ref7}
Pedregosa, F., Varoquaux, G., Gramfort, A., Michel, V., Thirion, B., Grisel, O., Blondel, M., Prettenhofer, P., Weiss, R. \& Dubourg, V. Scikit-learn: Machine learning in Python.  {\em Journal Of Machine Learning Research}. \textbf{12}, 2825--2830 (201)

\end{thebibliography}
